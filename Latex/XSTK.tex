\documentclass[12pt]{article}
%Page set up
\setlength{\topmargin}{-1in}
\setlength{\textwidth}{7in}
\setlength{\oddsidemargin}{-.25in}
\setlength{\textheight}{9.5in}
%Shortcuts
\def\it{\item}
\def\ul#1{\underline{#1}} \def\spul{\underline{\hspace{.75in}}} \def\ol#1{\overline{#1}}
\def\sspul{\ \underline{\hspace{.25in}} \ } \def\blank{\ul{\h{1.5}}}
\def\v#1{\vspace{#1in}} \def\h#1{\hspace{#1in}}
\def\be{\begin{enumerate}} \def\ee{\end{enumerate}}
\def\bc{\begin{center}} \def\ec{\end{center}}
\def\bd{\begin{description}} \def\ed{\end{description}}
\def\bt{\begin{tabular}} \def\et{\end{tabular}}
\def\lp{\left(} \def\rp{\right)} \def\abs#1{\vert #1 \vert}
\def\bar#1{\overline{#1}}
%General Math
\def\dis{\displaystyle}
\def\Frac#1#2{\displaystyle{\frac{#1}{#2}}}
\def\rta{\rightarrow}
\def\inv#1{{#1}^{-1}} % inverse of arg 
\def\th{\theta} \def\al{\alpha} \def\ba{\beta} \def\ga{\gamma}
\def\R{\mathbb{R}} \def\Q{\mathbb{Q}} \def\N{\mathbb{N}} \def\Z{\mathbb{Z}} \def\P{\mathbb{P}}
\def\X{\mathbb{X}} \def\Y{\mathbb{Y}} \def\U{\mathbb{U}} \def\E{\mathbb{E}} \def\C{\mathbb{C}}
\def\F{\mathbb{F}}
\def\mtrx#1{\begin{pmatrix}
#1_{11} & #1_{12} & \cdots & #1_{1n} \\
#1_{21} & #1_{22} & \cdots & #1_{2n} \\
\vdots & \vdots & \ddots & \vdots \\
#1_{m1} & #1_{m2} & \cdots & #1_{mn}
\end{pmatrix} }
\def\Idd{\begin{pmatrix} 1 & 0 \\ 0 & 1 \end{pmatrix}}
\def\Iddd{\begin{pmatrix} 1 & 0 & 0 \\ 0 & 1 & 0 \\ 0 & 0 & 1 \end{pmatrix}}
\def\colt#1#2{\lp \begin{array}{rr} #1 \\ #2 \end{array} \rp}
\def\colth#1#2#3{\lp \begin{array}{rrr} #1 \\ #2 \\ #3 \end{array} \rp}
\def\colf#1#2#3#4{\lp \begin{array}{rrrr} #1 \\ #2 \\ #3 \\ #4 \end{array} \rp}
\def\colp#1{\begin{pmatrix} #1_1 \\ #1_2 \\ \vdots \\ #1_p \end{pmatrix}}
\def\colm#1{\begin{pmatrix} #1_1 \\ #1_2 \\ \vdots \\ #1_m \end{pmatrix}}
\def\coln#1{\begin{pmatrix} #1_1 \\ #1_2 \\ \vdots \\ #1_n \end{pmatrix}}
\def\rowp#1{\begin{pmatrix} #1_1 & #1_2 & \cdots & #1_p \end{pmatrix}}
\def\rowt#1#2{\begin{pmatrix} #1 & #2 \end{pmatrix}}
\def\rowth#1#2#3{\begin{pmatrix} #1 & #2 & #3 \end{pmatrix}}
\def\rowf#1#2#3#4{\begin{pmatrix} #1 & #2 & #3 & #4 \end{pmatrix}}
\def\bfourt{\begin{tabbing}
xxxxxxxxxxxxxxxxxx \=
xxxxxxxxxxxxxxxxxx \=
xxxxxxxxxxxxxxxxxx \=
xxxxxxxxxxxxxxxxxx \kill}
\def\bttt{\begin{tabbing}
xxxxxxxxxxxxxxxxxxxxxxxxxxxxx \=
xxxxxxxxxxxxxxxxxxxxxxxxxxxxx \=
xxxxxxxxxxxxxxxxxxxxxxxxxxxxx \kill}
\def\btt{\begin{tabbing}
xxxxxxxxxxxxxxxxxxxxxxxxxxxxxxxxxxxxxxxxx \=
xxxxxxxxxxxxxxxxxxxxxxxxxxxxx \kill}
\def\etb{\end{tabbing}}
% cú pháp: Định nghĩa i: nội dung
\newtheorem{thm}{Định nghĩa}
\newtheorem{ex}[thm]{VD}
\newenvironment{linsys}[2][m]{%
\setlength{\arraycolsep}{.1111em} % p. 170 TeXbook; a medmuskip
\begin{array}[#1]{@{}*{#2}{rc}r@{}} 
}{%
\end{array}}

\usepackage{amssymb,amsmath,pstricks,framed}
\usepackage[vietnamese]{babel}
\usepackage{amsmath}
\usepackage[utf8]{inputenc}
\usepackage[T1]{fontenc}
\usepackage{graphicx}
\usepackage{color}
\usepackage{hyperref}

\setcounter{page}{1}
\begin{document}

% những nội dung cơ bản về môn xác suất thống kê
% tiêu đề, không chứa ngày tháng
\title{Xác suất thống kê}
\date{}
\maketitle
% nội dung
% phần 1: Công thức xác suất
% phần 2: Hàm mật độ xác suất, hàm phân phối xác suất
% phần 3: Phân phối nhị thức
% phần 4: Phân phối poisson
% phần 5: Phân phối chuẩn
% phần 6: Các dạng bài toán xác suất thống kê
% phần 7: Bảng giá trị

\section{Công thức xác suất}
\subsection{Kiến thức cơ bản}
% định nghĩa : P(A) = xác suất của biến cố A
\begin{thm}
$P(A)$ được gọi là xác suất của biến cố A
\end{thm}

\begin{thm}
Xác suất biến cố đối của A là xác suất không xảy ra A
\begin{equation}
P(\bar{A}) = 1 - P(A)
\end{equation}
\end{thm}
\begin{thm}
Nếu A B độc lập
\begin{equation}
P(A + B) = P(A) + P(B) 
\end{equation}
\begin{equation}
    P(AB) = P(A)P(B)
\end{equation}
\end{thm}

\begin{thm}
Nếu A B không độc lập
\begin{equation}
P(A + B) = P(A) + P(B) - P(AB)
\end{equation}
\begin{equation}
    P(AB) = P(A)P(B) - P(A + B)
\end{equation}
\end{thm}

\subsection{Xác suất có điều kiện}
\begin{thm}
Xác suất của biến cố A khi biết biến cố B đã xảy ra
\begin{equation}
P(A|B) = \frac{P(AB)}{P(B)}
\end{equation}
Từ đó có công thức
\begin{equation}
P(AB) = P(A|B)P(B) = P(B|A)P(A)
\end{equation}
Với 3 biến cố A B C
\begin{equation}
P(ABC) = P(A|BC)P(BC) = P(B|AC)P(AC) = P(C|AB)P(AB)
\end{equation}
\end{thm}

\begin{thm}
Công thức xác suất toàn phần
\begin{equation}
P(F) = \sum_{i=1}^n P(F|A_i)P(A_i)
\end{equation}
Hay ta có thể viết lại
\begin{equation}
P(F) = P(F|A_1)P(A_1) + P(F|A_2)P(A_2) + \cdots + P(F|A_n)P(A_n)
\end{equation}
\end{thm}

\begin{thm}
Công thức Bayes
\begin{equation}
P(A_i|F) = \frac{P(F|A_i)P(A_i)}{P(F)}
\end{equation}
\end{thm}

\subsection{Công thức Bernoulli}

\begin{thm}
Xác suất của biến cố A xảy ra k lần trong n lần thử là:
\begin{equation}
P(A) = C_n^k p^k (1-p)^{n-k}
\end{equation}
\end{thm}

\section{Hàm phân phối xác suất, hàm mật độ xác suất}
\subsection{Hàm phân phối xác suất}
\begin{thm}
Hàm phân phối xác suất của biến ngẫu nhiên X có dạng
\end{thm}
% table of probability distribution function
\begin{tabular}{|c|c|}
    \hline
    $X$ & $x_1$, $x_2$, $\cdots$, $x_n$ \\
    \hline
    $P(X)$ & $p_1$, $p_2$, $\cdots$, $p_n$ \\
    \hline
  \end{tabular}

\subsection{Hàm phân mât}

\end{document}